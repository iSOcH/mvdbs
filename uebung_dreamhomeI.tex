\documentclass[11pt,a4paper,parskip=half]{scrartcl}
\usepackage{ngerman}
\usepackage[utf8]{inputenc}
\usepackage[colorlinks=false,pdfborder={0 0 0}]{hyperref}
\usepackage{graphicx}
\usepackage{caption}
\usepackage{longtable}
\usepackage{float}
\usepackage{textcomp}
\usepackage{geometry}
\usepackage{amsmath}
\geometry{a4paper, left=30mm, right=25mm, top=30mm, bottom=35mm} 
\usepackage{listings}
\lstset{breaklines=true, breakatwhitespace=true, basicstyle=\scriptsize, numbers=left}
\title{mvdbs: Übung DreamHome I}
\author{Tobias Lerch, Yanick Eberle, Pascal Schwarz}
\begin{document}
\maketitle

\section{Simple predicates}
\subsection{Relation Filiale}
Die Relation Filiale enthält die Spalten FNr, Region und Bereich. Da FNr der Primärschlüssel der Relation ist, wird dieser nicht als simple predicate verwendet. Der Primärschlüssel ist eindeutig in der Relation und daher kein sinnvoller Wert für simple predicates. Die Werte in den Spalten Region und Bereich können jedoch zu folgenden simple predicates unterteilt werden.

p\textsubscript{1}: Region = 'Basel'\newline
p\textsubscript{2}: Region = 'Zürich'\newline
p\textsubscript{3}: Region = 'Solothurn'\newline
p\textsubscript{4}: Region = 'Aargau'\newline
\newline
p\textsubscript{5}: Bereich = 'Wohnungen'\newline
p\textsubscript{6}: Bereich = 'Büroräumlichkeiten'\newline
p\textsubscript{7}: Bereich = 'Häuser'\newline
p\textsubscript{8}: Bereich = 'Lager'\newline

\textbf{Vollständigkeit \& Minimalität: }
Die Vollständigkeit und Minimalität wurde eingehalten, da alle simple predicates relevant sind und nach deren Häufigkeit aufgeteilt werden können.\newline
Zugriff: Wohungen (sehr häufig),  Büroräumlichkeiten \& Lager (häufig), Häuser (selten).\newline
Relevanz: Basel (Wh, Hs), Zürich (Br, Wh, Hs, Lg), Solothurn (Br, Wh, Hs, Lg), Aargau (Wh, Hs, Lg).


\subsection{Relation Besitzer}
Die Relation Besitzer enthält die Spalten BNr, Name, PLZ und Ort. Wie bereits in der Relation Filiale wird auch hier der Primärschlüssel BNr nicht als simple predicate verwendet. Die Spalte Namen kann ebenfalls nicht sinnvoll verwendet werden da es eine zu grosse Menge an Fragmenten geben würde, welche zudem nicht den regionalen Filialen zugewiesen werden können. Eine Ortschaft hat nicht immer eine eindeutige PLZ (Bsp: Bern hat 3000-3030), weshalb sie ebenfalls nicht verwendet werden kann. Als simple predicate wird daher nur die PLZ benutzt. Da Basel, Solothurn und Aargau nah beisammen liegen, ist eine optimale Aufteilung der PLZ-Bereiche nicht sehr einfach. Wir haben uns für folgende Aufteilung entschieden.

b\textsubscript{1}: PLZ BETWEEN 1000 AND 3999 (Region Solothurn)\newline
b\textsubscript{2}: PLZ BETWEEN 4000 AND 4499 (Region Basel)\newline
b\textsubscript{3}: PLZ BETWEEN 4500 AND 4999 (Region Solothurn)\newline
b\textsubscript{4}: PLZ BETWEEN 5000 AND 6999 (Region Aargau)\newline
b\textsubscript{5}: PLZ BETWEEN 7000 AND 9999 (Region Zürich)\newline

\textbf{Vollständigkeit \& Minimalität: } Die Vollständigkeit und Minimalität ist gegeben, da alle simple predicates relevant sind und die Wahrscheinlichkeit eines Besitzer bei jeder Postleizahl gleich gross ist.

\section{Minterm predicates}
\subsection{Relation Filiale}
Da die minterm predicates exponentiell in der Anzahl der simple predicates sind, haben wir bei der Relation Filiale 2\textsuperscript{8} = 256 Möglichkeiten, minterm perdicates zu erstellen. Davon können wir aber nicht sinnvolle predicates ausschliessen. Da eine Filiale jeweils nur in einer Region tätig ist, können wir die minterm predicates auf folgende Kombinationen reduzieren.

\begin{align*}
	p\textsubscript{1} \wedge \neg p\textsubscript{2} \wedge \neg p\textsubscript{3} \wedge \neg p\textsubscript{4} \wedge ...\\
	\neg p\textsubscript{1} \wedge p\textsubscript{2} \wedge \neg p\textsubscript{3} \wedge \neg p\textsubscript{4} \wedge ...\\
	\neg p\textsubscript{1} \wedge \neg p\textsubscript{2} \wedge p\textsubscript{3} \wedge \neg p\textsubscript{4} \wedge ...\\
	\neg p\textsubscript{1} \wedge \neg p\textsubscript{2} \wedge \neg p\textsubscript{3} \wedge p\textsubscript{4} \wedge ...\\
\end{align*}

Zusätzlich wissen wir, dass nicht alle Bereiche in jeder Filiale angeboten werden. Dadurch würden ebenfalls Minterme wegfallen. Das bedeutet jedoch, dass zum Beispiel in der Filiale Basel nie Lager oder Büroräumlichkeiten angeboten werden können. Würde ein solcher Datensatz trotzdem erfasst, wäre die Vollständigkeit der Relation nicht mehr gegeben. Aus diesem Grund werden diese Minterme beibehalten, was uns die Anpassung der vertretenen Bereiche in den Filialen zu einem späteren Zeitpunkt ermöglicht. Zusätzlich könnnen alle Minterme, welche keine Bereiche enthalten ebenfalls entfernt werden ($\neg p\textsubscript{5} \wedge \neg p\textsubscript{6} \wedge \neg p\textsubscript{7} \wedge \neg p\textsubscript{8}$).

Durch das Entfernen von nicht relevanten Mintermen haben wir pro Filiale noch 2\textsuperscript{4} - 1 = 15 Möglichkeiten. Das reduziert die Anzahl minterm predicates der Relation Filiale von 256 auf 60.

\subsubsection{Sinnvolle minterm predicates Region Basel (p1)}
\begin{align*}
p_1 \wedge p_5 \wedge p_6 \wedge p_7 \wedge p_8\\
p_1 \wedge p_5 \wedge p_6 \wedge p_7 \wedge \neg p_8\\
p_1 \wedge p_5 \wedge p_6 \wedge \neg p_7 \wedge p_8\\
p_1 \wedge p_5 \wedge p_6 \wedge \neg p_7 \wedge \neg p_8\\
p_1 \wedge p_5 \wedge \neg p_6 \wedge \neg p_7 \wedge p_8\\
p_1 \wedge p_5 \wedge \neg p_6 \wedge \neg p_7 \wedge \neg p_8\\
p_1 \wedge p_5 \wedge \neg p_6 \wedge  p_7 \wedge p_8\\
p_1 \wedge p_5 \wedge \neg p_6 \wedge  p_7 \wedge \neg p_8\\
p_1 \wedge \neg p_5 \wedge \neg p_6 \wedge \neg p_7 \wedge p_8\\
p_1 \wedge \neg p_5 \wedge \neg p_6 \wedge  p_7 \wedge p_8\\
p_1 \wedge \neg p_5 \wedge \neg p_6 \wedge  p_7 \wedge \neg p_8\\
p_1 \wedge \neg p_5 \wedge  p_6 \wedge \neg p_7 \wedge p_8\\
p_1 \wedge \neg p_5 \wedge  p_6 \wedge \neg p_7 \wedge \neg p_8\\
p_1 \wedge \neg p_5 \wedge  p_6 \wedge  p_7 \wedge  p_8\\
p_1 \wedge \neg p_5 \wedge  p_6 \wedge  p_7 \wedge \neg p_8\\
\end{align*}

\subsubsection{Sinnvolle minterm predicates Region Zürich (p2)}
\begin{align*}
p_2 \wedge p_5 \wedge p_6 \wedge p_7 \wedge p_8\\
p_2 \wedge p_5 \wedge p_6 \wedge p_7 \wedge \neg p_8\\
p_2 \wedge p_5 \wedge p_6 \wedge \neg p_7 \wedge p_8\\
p_2 \wedge p_5 \wedge p_6 \wedge \neg p_7 \wedge \neg p_8\\
p_2 \wedge p_5 \wedge \neg p_6 \wedge \neg p_7 \wedge p_8\\
p_2 \wedge p_5 \wedge \neg p_6 \wedge \neg p_7 \wedge \neg p_8\\
p_2 \wedge p_5 \wedge \neg p_6 \wedge  p_7 \wedge p_8\\
p_2 \wedge p_5 \wedge \neg p_6 \wedge  p_7 \wedge \neg p_8\\
p_2 \wedge \neg p_5 \wedge \neg p_6 \wedge \neg p_7 \wedge p_8\\
p_2 \wedge \neg p_5 \wedge \neg p_6 \wedge  p_7 \wedge p_8\\
p_2 \wedge \neg p_5 \wedge \neg p_6 \wedge  p_7 \wedge \neg p_8\\
p_2 \wedge \neg p_5 \wedge  p_6 \wedge \neg p_7 \wedge p_8\\
p_2 \wedge \neg p_5 \wedge  p_6 \wedge \neg p_7 \wedge \neg p_8\\
p_2 \wedge \neg p_5 \wedge  p_6 \wedge  p_7 \wedge  p_8\\
p_2 \wedge \neg p_5 \wedge  p_6 \wedge  p_7 \wedge \neg p_8\\
\end{align*}

\subsubsection{Sinnvolle minterm predicates Region Solothurn (p3)}
\begin{align*}
p_3 \wedge p_5 \wedge p_6 \wedge p_7 \wedge p_8\\
p_3 \wedge p_5 \wedge p_6 \wedge p_7 \wedge \neg p_8\\
p_3 \wedge p_5 \wedge p_6 \wedge \neg p_7 \wedge p_8\\
p_3 \wedge p_5 \wedge p_6 \wedge \neg p_7 \wedge \neg p_8\\
p_3 \wedge p_5 \wedge \neg p_6 \wedge \neg p_7 \wedge p_8\\
p_3 \wedge p_5 \wedge \neg p_6 \wedge \neg p_7 \wedge \neg p_8\\
p_3 \wedge p_5 \wedge \neg p_6 \wedge  p_7 \wedge p_8\\
p_3 \wedge p_5 \wedge \neg p_6 \wedge  p_7 \wedge \neg p_8\\
p_3 \wedge \neg p_5 \wedge \neg p_6 \wedge \neg p_7 \wedge p_8\\
p_3 \wedge \neg p_5 \wedge \neg p_6 \wedge  p_7 \wedge p_8\\
p_3 \wedge \neg p_5 \wedge \neg p_6 \wedge  p_7 \wedge \neg p_8\\
p_3 \wedge \neg p_5 \wedge  p_6 \wedge \neg p_7 \wedge p_8\\
p_3 \wedge \neg p_5 \wedge  p_6 \wedge \neg p_7 \wedge \neg p_8\\
p_3 \wedge \neg p_5 \wedge  p_6 \wedge  p_7 \wedge  p_8\\
p_3 \wedge \neg p_5 \wedge  p_6 \wedge  p_7 \wedge \neg p_8\\
\end{align*}

\subsubsection{Sinnvolle minterm predicates Region Aargau (p4)}
\begin{align*}
p_4 \wedge p_5 \wedge p_6 \wedge p_7 \wedge p_8\\
p_4 \wedge p_5 \wedge p_6 \wedge p_7 \wedge \neg p_8\\
p_4 \wedge p_5 \wedge p_6 \wedge \neg p_7 \wedge p_8\\
p_4 \wedge p_5 \wedge p_6 \wedge \neg p_7 \wedge \neg p_8\\
p_4 \wedge p_5 \wedge \neg p_6 \wedge \neg p_7 \wedge p_8\\
p_4 \wedge p_5 \wedge \neg p_6 \wedge \neg p_7 \wedge \neg p_8\\
p_4 \wedge p_5 \wedge \neg p_6 \wedge  p_7 \wedge p_8\\
p_4 \wedge p_5 \wedge \neg p_6 \wedge  p_7 \wedge \neg p_8\\
p_4 \wedge \neg p_5 \wedge \neg p_6 \wedge \neg p_7 \wedge p_8\\
p_4 \wedge \neg p_5 \wedge \neg p_6 \wedge  p_7 \wedge p_8\\
p_4 \wedge \neg p_5 \wedge \neg p_6 \wedge  p_7 \wedge \neg p_8\\
p_4 \wedge \neg p_5 \wedge  p_6 \wedge \neg p_7 \wedge p_8\\
p_4 \wedge \neg p_5 \wedge  p_6 \wedge \neg p_7 \wedge \neg p_8\\
p_4 \wedge \neg p_5 \wedge  p_6 \wedge  p_7 \wedge  p_8\\
p_4 \wedge \neg p_5 \wedge  p_6 \wedge  p_7 \wedge \neg p_8\\
\end{align*}

\subsection{Definition Fragmente Relation Filiale}

FILIALE$_1$: $\sigma$\textsubscript{}\\


\subsection{Relation Besitzer}
In der Relation Besitzer haben wir 5 verschiedene simple predicates, womit wir 2\textsuperscript{5} = 32 minterm predicates erhalten. Da ein Besitzer nur einen Wohnort bzw. eine PLZ hat, kann nur genau ein simple predicate zutreffen. Das bedeutet unsere minterm predicates können von 32 auf 5 Möglichkeiten reduziert werden.
\begin{align*}
b_1 \wedge \neg b_2 \wedge \neg b_3 \wedge \neg b_4 \wedge \neg b_5\\
\neg b_1 \wedge  b_2 \wedge \neg b_3 \wedge \neg b_4 \wedge \neg b_5\\
\neg b_1 \wedge \neg b_2 \wedge  b_3 \wedge \neg b_4 \wedge \neg b_5\\
\neg b_1 \wedge \neg  b_2 \wedge \neg b_3 \wedge  b_4 \wedge \neg b_5\\
\neg b_1 \wedge \neg b_2 \wedge \neg b_3 \wedge \neg b_4 \wedge b_5\\
\end{align*}

\subsection{Definition Fragmente Relation Besitzer}

BESITZER\textsubscript{1}: $\sigma$\textsubscript{PLZ  BETWEEN 1000 AND 3999 OR BETWEEN 4500 AND 4999} (BESITZER)\\
BESITZER\textsubscript{2}: $\sigma$\textsubscript{PLZ  BETWEEN 4000 AND 4499} (BESITZER)\\
BESITZER\textsubscript{3}: $\sigma$\textsubscript{PLZ  BETWEEN 5000 AND 6999} (BESITZER)\\
BESITZER\textsubscript{4}: $\sigma$\textsubscript{PLZ  BETWEEN 7000 AND 9999} (BESITZER)\\

\end{document}