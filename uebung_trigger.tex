\documentclass[11pt,a4paper,parskip=half]{scrartcl}
\usepackage{ngerman}
\usepackage[utf8]{inputenc}
\usepackage[colorlinks=false,pdfborder={0 0 0}]{hyperref}
\usepackage{graphicx}
\usepackage{caption}
\usepackage{longtable}
\usepackage{float}
\usepackage{textcomp}
\usepackage{geometry}
\geometry{a4paper, left=30mm, right=25mm, top=30mm, bottom=35mm} 
\usepackage{listings}
\lstset{breaklines=true, breakatwhitespace=true, basicstyle=\scriptsize, numbers=left}
\title{mvdbs: Übung Trigger}
\author{Tobias Lerch, Yanick Eberle, Pascal Schwarz}
\begin{document}
\maketitle

\section{Aufgabe 1 - Event Logging}
\subsection{Lösungsidee}
Wir erstellen einen Trigger, welcher bei den SQL Statements, die potenziell Änderungen an der Tabelle \glqq{}Ausleihen\grqq{} bewirken, ausgelöst wird. Wie in der Aufgabenstellung beschrieben sind dies die folgenden SQL-Befehle:

\begin{itemize}
	\item{INSERT}
	\item{UPDATE}
	\item{DELETE}
\end{itemize}

Aufgrund der Anforderung sowohl die alten wie auch die neuen Werte zu protokollieren, muss unser Trigger jeweils vor dem Statement ausgeführt werden. In einem solchen Trigger haben wir Zugriff auf die neuen Werte.

\subsection{Tabelle Ausleihen\_Log}
Die Log-Tabelle enthält einen eigenen Primary Key (Number(6,0)). Die geforderten Angaben (User, welcher die Änderung vorgenommen hat (VARCHAR2(20)), Art der Änderung (VARCHAR(3)) sowie Zeitpunkt der Änderung (TIMESTAMP(6))) werden jeweils in einem Attribut abgelegt.

Zusätzlich erhält die Log-Tabelle für jedes Attribut der Tabelle Ausleihen zwei Attribute. In \emph{Feldname}\_old wird der Wert vor der Änderung, in \emph{Feldname}\_new der Wert nach der Änderung festgehalten. Diese Attribute haben jeweils den selben Datentyp wie das jeweilige Attribut in der Tabelle \glqq{}Ausleihen\grqq{}.

\subsection{Trigger für Protokollierung}
Da ein Statement grundsätzlich mehrere Zeilen der Tabelle \glqq{}auf einmal\grqq{} verändern kann, muss der Trigger mit der Granularität \glqq{}FOR EACH ROW\grqq{} definiert werden.

Wie bei der Lösungsidee bereits beschrieben wird der Trigger vor den Events INSERT, UPDATE und DELETE ausgelöst.
Damit wir uns nicht um den Primärschlüssel kümmern müssen, erstellen wir eine Sequenz und erhöhen diese bei jedem Eintrag in die Tabelle  ausleihen\_log um eins.


\subsection{SQL Statements}
\subsubsection{Ausleihen\_Log}
\begin{lstlisting}
CREATE TABLE AUSLEIHEN_LOG
(
  LOG_ID NUMBER(6, 0) NOT NULL 
, CHANGE_USER VARCHAR2(20) NOT NULL 
, CHANGE_DATE TIMESTAMP(6) NOT NULL 
, CHANGE_TYPE VARCHAR2(3) NOT NULL
, MNR_OLD VARCHAR2(4)
, MNR_NEW VARCHAR2(4)
, DVDNR_OLD NUMBER(6, 0)
, DVDNR_NEW NUMBER(6, 0)
, DATUM_OLD DATE
, DATUM_NEW DATE
, RUECKGABE_OLD DATE 
, RUECKGABE_NEW DATE 
, CONSTRAINT AUSLEIHEN_LOG_PK PRIMARY KEY (log_id) ENABLE 
);
\end{lstlisting}

\subsubsection{Trigger}
\begin{lstlisting}
  CREATE OR REPLACE TRIGGER ausleihen_logger 
  BEFORE UPDATE OR INSERT OR DELETE ON ausleihen
  FOR EACH ROW
  DECLARE
    manipulation varchar2(3);
    new_log_id number(6,0);
  BEGIN
    if inserting then
      manipulation := 'INS';
    elsif deleting then
      manipulation := 'DEL';
    elsif updating then
      manipulation := 'UPD';
    else 
      manipulation := 'ERR';
    end if;
    
    SELECT seq_ausleih_log_id.nextval INTO new_log_id FROM dual; 
    INSERT INTO ausleihen_log (log_id, change_user, change_date, change_type, mnr_old, mnr_new, dvdnr_old, dvdnr_new, datum_old, datum_new, rueckgabe_old, rueckgabe_new) 
    VALUES (new_log_id, user, sysdate, manipulation, :old.mnr, :new.mnr, :old.dvdnr, :new.dvdnr, :old.datum, :new.datum, :old.rueckgabe, :new.rueckgabe);
END;
\end{lstlisting}

\subsection{Tests}
Um zu überprüfen, ob der von uns erstellte Trigger ausleihen\_logger richtig funktioniert und alle verlangten Informationen in der Tabelle ausleihen\_log eingetragen sind, führen wir für jede potentielle Änderung einen Tests durch.

\subsubsection{Test INSERT}

Mit folgendem SQL-Befehl fügen wir Daten in die Tabelle Ausleihen ein:

\begin{lstlisting}
INSERT INTO ausleihen (mnr,dvdnr,datum,rueckgabe) 
SELECT 'M005',468123,'01.01.2000',null FROM dual
UNION ALL SELECT 'M005',183669,'01.01.2000',null FROM dual;
\end{lstlisting}

In der Tabelle ausleihen sind nun die eingefügten Daten in der 4. und 5. Zeile ersichtlich:

\begin{lstlisting}
MNR	DVDNR	 DATUM		RUECKGABE
---- 	------	-------- 	---------
M001	468123 	01.01.99	
M005	468123	01.01.00	
M005	183669	01.01.00	
M002	158234 	19.07.07	21.07.07  
M004	158234 	02.08.07	04.08.07  
M003	269260 	05.01.08	
M003	199004 	05.01.08	
M001	310094 	22.11.07	27.11.07  
M001	468123 	19.01.08	
M002	183669 	30.11.07	01.12.07  
M004	183669 	27.12.07	03.01.08  
M005	183669 	15.01.08	
M001	183669	01.01.99	
\end{lstlisting}

In der Tabelle ausleihen\_log wurden die Daten mit der LOG\_ID 22 und 23 eingefügt (Zeile 3 und 4):

\begin{lstlisting}
LOG_ID	CHANGE_USER	CHANGE_DATE			CHANGE_TYPE	
------	-----------	---------------------------	----------		
22	MVDBS10		24.02.13 15:19:50,000000000	INS					
23	MVDBS10		24.02.13 15:19:50,000000000	INS					
\end{lstlisting}

\begin{lstlisting}
MNR_OLD	MNR_NEW		DVDNR_OLD	DVDNR_NEW	DATUM_OLD	DATUM_NEW	
-------	-------		----------	----------	---------	--------- 	
	M005				468123				01.01.00					
	M005				183669				01.01.00					
\end{lstlisting}
\begin{lstlisting}
RUECKGABE_OLD	RUECKGABE_NEW
-------------	-------------	
						.
						.
\end{lstlisting}

Da es sich hier um ein INSERT handelt, sind alle \_old Felder leer. Der Trigger hat also funktioniert.

\subsubsection{Test UPDATE}
Mit folgendem SQL-Befehl ändern wir Daten in der Tabelle Ausleihen:

\begin{lstlisting}
UPDATE ausleihen
set mnr = 'M004'
WHERE mnr = 'M005' AND 
      datum LIKE '01.01.00';
\end{lstlisting}

In der Tabelle ausleihen sind nun die geänderten Daten in der 4. und 5. Zeile ersichtlich:

\begin{lstlisting}
MNR	DVDNR	 DATUM		RUECKGABE
---- 	------	-------- 	---------
M001	468123 	01.01.99	
M004	468123	01.01.00	
M004	183669	01.01.00	
M002	158234 	19.07.07	21.07.07  
M004	158234 	02.08.07	04.08.07  
M003	269260 	05.01.08	
M003	199004 	05.01.08	
M001	310094 	22.11.07	27.11.07  
M001	468123 	19.01.08	
M002	183669 	30.11.07	01.12.07  
M004	183669 	27.12.07	03.01.08  
M005	183669	15.01.08	
M001	183669	01.01.99	
\end{lstlisting}

In der Tabelle ausleihen\_log wurden die Daten mit der LOG\_ID 24 und 25 eingefügt (Zeile 5 und 6):

\begin{lstlisting}
LOG_ID	CHANGE_USER	CHANGE_DATE			CHANGE_TYPE	
------	-----------	---------------------------	----------		
22	MVDBS10		24.02.13 15:19:50,000000000	INS					
23	MVDBS10		24.02.13 15:19:50,000000000	INS					
24	MVDBS10		24.02.13 15:34:44,000000000	UPD					
25	MVDBS10		24.02.13 15:34:44,000000000	UPD					
\end{lstlisting}

\begin{lstlisting}
MNR_OLD	MNR_NEW		DVDNR_OLD	DVDNR_NEW	DATUM_OLD	DATUM_NEW	
-------	-------		----------	----------	---------	--------- 	
	M005				468123				01.01.00					
	M005				183669				01.01.00					
M005	M004		468123		468123		01.01.00	01.01.00				
M005	M004		183669		183669		01.01.00	01.01.00				
\end{lstlisting}
\begin{lstlisting}
RUECKGABE_OLD	RUECKGABE_NEW
-------------	-------------	
						.
						.
						.
						.
\end{lstlisting}

Nun sieht man, dass die \_old Felder ebenfalls ausgefüllt sind mit den Werten vor dem Update. Der Trigger hat also funktioniert.

\subsubsection{Test DELETE}
Mit folgendem SQL-Befehl löschen wir Daten in der Tabelle Ausleihen:

\begin{lstlisting}
DELETE FROM ausleihen
WHERE mnr = 'M004' AND
      datum like '01.01.00';
\end{lstlisting}

In der Tabelle ausleihen sind nun die gelöschten Daten nicht mehr ersichtlich:

\begin{lstlisting}
MNR	DVDNR	 DATUM		RUECKGABE
---- 	------	-------- 	---------
M001	468123 	01.01.99	
M002	158234 	19.07.07	21.07.07  
M004	158234 	02.08.07	04.08.07  
M003	269260 	05.01.08	
M003	199004 	05.01.08	
M001	310094 	22.11.07	27.11.07  
M001	468123 	19.01.08	
M002	183669 	30.11.07	01.12.07  
M004	183669 	27.12.07	03.01.08  
M005	183669 	15.01.08	
M001	183669 	01.01.99	
\end{lstlisting}

In der Tabelle ausleihen\_log wurden die Daten mit der LOG\_ID 26 und 27 eingefügt (Zeile 7 und 8):

\begin{lstlisting}
LOG_ID	CHANGE_USER	CHANGE_DATE			CHANGE_TYPE	
------	-----------	---------------------------	----------		
22	MVDBS10		24.02.13 15:19:50,000000000	INS					
23	MVDBS10		24.02.13 15:19:50,000000000	INS					
24	MVDBS10		24.02.13 15:34:44,000000000	UPD					
25	MVDBS10		24.02.13 15:34:44,000000000	UPD					
26	MVDBS10		24.02.13 15:41:14,000000000	DEL					
27	MVDBS10		24.02.13 15:41:14,000000000	DEL					
\end{lstlisting}

\begin{lstlisting}
MNR_OLD	MNR_NEW		DVDNR_OLD	DVDNR_NEW	DATUM_OLD	DATUM_NEW	
-------	-------		----------	----------	---------	--------- 	
	M005				468123				01.01.00					
	M005				183669				01.01.00					
M005	M004		468123		468123		01.01.00	01.01.00				
M005	M004		183669		183669		01.01.00	01.01.00				
M004			468123				01.01.00					
M004			183669				01.01.00					
\end{lstlisting}
\begin{lstlisting}
RUECKGABE_OLD	RUECKGABE_NEW
-------------	-------------	
						.
						.
						.
						.
						.
						.
\end{lstlisting}

Da es sich nun um ein DELETE-Statement handelt, gibt es keine neuen Werte, daher sind die \_new Felder leer. Der Trigger hat also funktioniert.

\section{Aufgabe 2 - Referential Integrity}
\subsection{Lösungsidee / Vorbereitung}
Für das Verschieben der Tabelle \glqq{}Filme\grqq{} muss der Foreign Key Constraint \glqq{}DK\_FM\_FK\grqq{} auf der Tabelle \glqq{}DVDKopien\grqq{} zunächst entfernt werden. Ansonsten kann die Tabelle nicht entfernt werden.

Als nächster Schritt wird ein Database-Link auf dem Server telesto (dort sind alle Tabellen ausser Filme) erstellt:

\begin{lstlisting}
CREATE DATABASE LINK orion.helios.fhnw.ch
CONNECT TO mvdbs10 identified by mvdbs10
USING 'orion';
\end{lstlisting}

Damit die entfernte Tabelle so benutzt werden kann als wäre sie auf diesem Server wird noch ein SYNONYM erstellt:
\begin{lstlisting}
CREATE SYNONYM filme FOR filme@orion.helios.fhnw.ch;
\end{lstlisting}

Wir brauchen einen entsprechenden Link auch von der anderen Seite her:

\begin{lstlisting}
CREATE DATABASE LINK telesto.janus.fhnw.ch
CONNECT to mvdbs10 identified by mvdbs10
USING 'telesto';
\end{lstlisting}

Auch hier erstellen wir ein SYNONYM
\begin{lstlisting}
CREATE SYNONYM dvdkopien FOR dvdkopien@telesto.janus.fhnw.ch;
\end{lstlisting}


\subsection{Entwurf der Trigger}
In den folgenden Fällen muss unser Trigger eingreifen:
\begin{enumerate}
	\item{Löschen eines Datensatzes aus Filme, auf den sich noch mindestens ein Datensatz aus DVDKopien bezieht. Wird vor dem Event DELETE auf der Tabelle Filme für jede Zeile angewendet. Sollte sich noch ein Datensatz aus DVDKopien auf den zu löschenden Film-Eintrag beziehen, so wird eine Exception geworfen.}
	\item{Ändern des Primärschlüssels (katalognr) eines Datensatzes aus Filme, auf den sich noch mindestens ein Datensatz aus DVDKopien bezieht. Der Trigger muss vor dem Event UPDATE auf der Spalte katalognr wiederum für jede betroffene Zeile angewendet werden. Auch hier soll eine Exception geworfen werden, falls einer der alten Werte in DVDKopien benutzt wurde.}
	\item{Einfügen eines Datensatzes in DVDKopien - die angegebene katalognr muss in Filme existieren. Um dies zu prüfen verwenden wir einen BEFORE INSERT-Trigger, welcher wiederum pro Zeile angestossen wird.}
	\item{Änderung von FK in DVDKopien, auch der neue Wert muss in Filme existieren. Dieser Fall wird mit einem Trigger geprüft, der vor dem Update auf der katalognr-Spalte der Tabelle DVDKopien gefeuert wird.}
\end{enumerate}

\subsection{SQL Trigger}
\subsubsection{Insert in DVDKopien}
\begin{lstlisting}
CREATE OR REPLACE TRIGGER dvdkopien_insert
  BEFORE INSERT on DVDKopien
  FOR EACH ROW
  
  DECLARE
    katalognr_found_count NUMBER(2,0) := 0;
    
  BEGIN
    SELECT count(katalognr) INTO katalognr_found_count FROM filme WHERE katalognr = :new.katalognr;
    
    dbms_output.put('count(katalognr) in filme: ');
    dbms_output.put_line(katalognr_found_count);
    
    if katalognr_found_count < 1 then
      raise_application_error(-20000, 'film mit angegebener katalognr existiert nicht');
    end if;
    
  END;
\end{lstlisting}

\subsubsection{Update in DVDKopien}
\begin{lstlisting}
CREATE OR REPLACE TRIGGER dvdkopien_update_katalognr
  BEFORE UPDATE OF katalognr ON DVDKopien
  FOR EACH ROW
  
  DECLARE
    katalognr_found_count NUMBER(2,0) := 0;
    
  BEGIN
    SELECT count(katalognr) INTO katalognr_found_count FROM filme WHERE katalognr = :new.katalognr;
    
    dbms_output.put('count(katalognr) in filme: ');
    dbms_output.put_line(katalognr_found_count);
    
    if katalognr_found_count < 1 then
      raise_application_error(-20000, 'film mit angegebener katalognr existiert nicht');
    end if;
    
  END;
\end{lstlisting}

\subsubsection{Update in Filme}
\begin{lstlisting}
CREATE OR REPLACE TRIGGER filme_update_katalognr
  BEFORE UPDATE OF katalognr ON filme
  FOR EACH ROW
  
  DECLARE
    katalognr_found_count NUMBER(6,0) := 0;
    
  BEGIN
    SELECT count(katalognr) INTO katalognr_found_count FROM dvdkopien WHERE katalognr = :old.katalognr;
    
    dbms_output.put('count(katalognr) in dvdkopien: ');
    dbms_output.put_line(katalognr_found_count);
    
    if katalognr_found_count > 0 then
      raise_application_error(-20000, 'es gibt noch dvdkopien dieses filmes');
    end if;
    
  END;
\end{lstlisting}

\subsubsection{Delete aus Filme}
\begin{lstlisting}
CREATE OR REPLACE TRIGGER filme_delete
  BEFORE DELETE on filme
  FOR EACH ROW
  
  DECLARE
    katalognr_found_count NUMBER(6,0) := 0;
    
  BEGIN
    SELECT count(katalognr) INTO katalognr_found_count FROM dvdkopien WHERE katalognr = :old.katalognr;
    
    dbms_output.put('count(katalognr) in dvdkopien: ');
    dbms_output.put_line(katalognr_found_count);
    
    if katalognr_found_count > 0 then
      raise_application_error(-20000, 'es gibt noch dvdkopien dieses filmes');
    end if;
    
  END;
\end{lstlisting}

\subsection{Tests}
Um zu überprüfen ob die Trigger richtig funktionieren, erstellen wir für jeden Fall zwei Tests, wobei einer der Tests das Erlaubte durchführt und der andere eine Exception verursacht.

\subsubsection{Test Insert in DVD Kopien}
Wir fügen in die Tabelle DVDKopien Daten mit einem vorhandenen FK katalognr ein:

\begin{lstlisting}
INSERT INTO dvdkopien (dvdnr,katalognr,fnr) 
SELECT 10001,2468,'F1' FROM dual 
UNION ALL SELECT 10002,2468,'F2' FROM dual;
\end{lstlisting}

Es wird keine Exception geworfen. In der Tabelle DVDKopien sind die eingefügten Daten nun ersichtlich (Zeile 3 und 4):

\begin{lstlisting}
     DVDNR  KATALOGNR FNR
---------- ---------- ---
     10001       2468 F1  
     10002       2468 F2  
    199004       2028 F1  
    468123       2028 F2  
    269260       1245 F1  
    183669       1245 F3  
    329270       1245 F4  
    178643       2239 F2  
    389653       2239 F4  
    158234       1062 F3  
    139558       2468 F2  
    469118       2468 F2  
    310094       1062 F2  
\end{lstlisting}

Für den zweiten Fall versuchen wir Daten einzufügen mit einem ungültigen FK katalognr:

\begin{lstlisting}
INSERT INTO dvdkopien (dvdnr,katalognr,fnr) 
SELECT 10003,2468,'F1' FROM dual 
UNION ALL SELECT 10004,9999,'F2' FROM dual;
\end{lstlisting}

Es wird eine Exception geworfen von unserem Trigger der die Meldung 'film mit angegebener katalognr existiert nicht' ausgibt.

\begin{lstlisting}
Fehler beim Start in Zeile 1 in Befehl:
INSERT INTO dvdkopien (dvdnr,katalognr,fnr) 
SELECT 10003,2468,'F1' FROM dual 
UNION ALL SELECT 10004,9999,'F2' FROM dual
Fehlerbericht:
SQL-Fehler: ORA-20000: film mit angegebener katalognr existiert nicht
ORA-06512: in "MVDBS10.DVDKOPIEN_INSERT", Zeile 11
ORA-04088: Fehler bei der Ausfuehrung von Trigger 'MVDBS10.DVDKOPIEN_INSERT'
20000. 00000 -  "%s"
*Cause:    The stored procedure 'raise_application_error'
           was called which causes this error to be generated.
*Action:   Correct the problem as described in the error message or contact
           the application administrator or DBA for more information.
\end{lstlisting}

Es wurden keine Daten in die Tabelle DVDKopien eingefügt:

\begin{lstlisting}
     DVDNR  KATALOGNR FNR
---------- ---------- ---
     10001       2468 F1  
     10002       2468 F2
    199004       2028 F1  
    468123       2028 F2  
    269260       1245 F1  
    183669       1245 F3  
    329270       1245 F4  
    178643       2239 F2  
    389653       2239 F4  
    158234       1062 F3  
    139558       2468 F2  
    469118       2468 F2  
    310094       1062 F2  
\end{lstlisting}

Wir haben ein Multi-Row-Insert mit zwei Zeilen durchgeführt, bei welchem eine Zeile korrekt war und die andere einen ungültigen FK katalognr enthalten hat. Da es sich um eine Transaktion handelt, darf alles oder gar nichts eingefügt werden. Daher ist es korrekt, dass das ganze Statement nicht ausgeführt wurde und somit keine Daten in die Tabelle DVDKopien eingefügt wurden.

\subsubsection{Test Update in DVD Kopien}
Nun werden wir einen bestehenden Eintrag ind er Tabelle DVDKopien anpassen und die katalognr mit einem anderen gültigen Wert ersetzen:

\begin{lstlisting}
UPDATE dvdkopien
SET katalognr = 2028
WHERE dvdnr LIKE '1000%';
\end{lstlisting}

Es wird keine Exception geworfen. In der Tabelle DVDKopien sind die geänderten Daten nun ersichtlich (Zeile 3 und 4):

\begin{lstlisting}
     DVDNR  KATALOGNR FNR
---------- ---------- ---
     10001       2028 F1  
     10002       2028 F2  
    199004       2028 F1  
    468123       2028 F2  
    269260       1245 F1  
    183669       1245 F3  
    329270       1245 F4  
    178643       2239 F2  
    389653       2239 F4  
    158234       1062 F3  
    139558       2468 F2  
    469118       2468 F2  
    310094       1062 F2  
\end{lstlisting}

Für den zweiten Fall versuchen wir den FK katalognr durch einen ungültigen FK zu ersetzen:

\begin{lstlisting}
UPDATE dvdkopien
SET katalognr = 9999
WHERE dvdnr LIKE '1000%';
\end{lstlisting}

Es wird eine Exception geworfen von unserem Trigger der die Meldung 'film mit angegebener katalognr existiert nicht' ausgibt.

\begin{lstlisting}
Fehler beim Start in Zeile 1 in Befehl:
UPDATE dvdkopien
SET katalognr = 9999
WHERE dvdnr LIKE '1000%'
Fehlerbericht:
SQL-Fehler: ORA-20000: film mit angegebener katalognr existiert nicht
ORA-06512: in "MVDBS10.DVDKOPIEN_UPDATE_KATALOGNR", Zeile 11
ORA-04088: Fehler bei der Ausfuehrung von Trigger 'MVDBS10.DVDKOPIEN_UPDATE_KATALOGNR'
20000. 00000 -  "%s"
*Cause:    The stored procedure 'raise_application_error'
           was called which causes this error to be generated.
*Action:   Correct the problem as described in the error message or contact
           the application administrator or DBA for more information.
\end{lstlisting}

Es wurden keine Daten in die Tabelle DVDKopien eingefügt:

\begin{lstlisting}
     DVDNR  KATALOGNR FNR
---------- ---------- ---
     10001       2028 F1  
     10002       2028 F2  
    199004       2028 F1  
    468123       2028 F2  
    269260       1245 F1  
    183669       1245 F3  
    329270       1245 F4  
    178643       2239 F2  
    389653       2239 F4  
    158234       1062 F3  
    139558       2468 F2  
    469118       2468 F2  
    310094       1062 F2  
\end{lstlisting}

\subsubsection{Test Update in Filme}
Wir ändern die katalognr eines bestehenden Filmes zu welchem keine DVDKopie existiert, auf welchen also keine Referenz vorhanden ist:

\begin{lstlisting}
UPDATE filme
SET katalognr = 1111
WHERE katalognr = 1672;
\end{lstlisting}

Es wird keine Exception geworfen. In der Tabelle Filme sind die geänderten Daten nun ersichtlich (Zeile 8):

\begin{lstlisting}
 KATALOGNR TITEL                          MINDESTALTER    GEBUEHR
---------- ------------------------------ ------------ ----------
      2028 Casablanca                                9        8.5 
      1245 Ocean's Eleven                           12        9.5 
      2239 A Space Odyssee                          12        7.5 
      1062 Pulp Fiction                             16        8.5 
      2588 The Pelican Brief                        12        8.5 
      1111 Erin Brockovich                           9        8.9 
      2468 Ratatouille                               6        7.5 
\end{lstlisting}

Für den zweiten Fall versuchen wir die katalognr eines bestehenden Filmes zu ändern, zu welchem eine DVDKopie existiert, auf welchen also eine Referenz vorhanden ist:

\begin{lstlisting}
UPDATE filme
SET katalognr = 2222
WHERE katalognr = 2468 OR 
      katalognr = 1111;
\end{lstlisting}

Es wird eine Exception geworfen von unserem Trigger der die Meldung 'es gibt noch dvdkopien dieses filmes' ausgibt.

\begin{lstlisting}
Fehler beim Start in Zeile 1 in Befehl:
UPDATE filme
SET katalognr = 2222
WHERE katalognr = 2468 OR 
      katalognr = 1111
Fehlerbericht:
SQL-Fehler: ORA-20000: es gibt noch dvdkopien dieses filmes
ORA-06512: in "MVDBS10.FILME_UPDATE_KATALOGNR", Zeile 11
ORA-04088: Fehler bei der Ausfuehrung von Trigger 'MVDBS10.FILME_UPDATE_KATALOGNR'
ORA-02063: vorherige 3 lines von ORION.HELIOS.FHNW.CH
20000. 00000 -  "%s"
*Cause:    The stored procedure 'raise_application_error'
           was called which causes this error to be generated.
*Action:   Correct the problem as described in the error message or contact
           the application administrator or DBA for more information.
\end{lstlisting}

Es wurden keine Daten in die Tabelle Filme geändert:

\begin{lstlisting}
 KATALOGNR TITEL                          MINDESTALTER    GEBUEHR
---------- ------------------------------ ------------ ----------
      2028 Casablanca                                9        8.5 
      1245 Ocean's Eleven                           12        9.5 
      2239 A Space Odyssee                          12        7.5 
      1062 Pulp Fiction                             16        8.5 
      2588 The Pelican Brief                        12        8.5 
      1111 Erin Brockovich                           9        8.9 
      2468 Ratatouille                               6        7.5 
\end{lstlisting}

Auch hier wird ein Update auf mehrer Datensätze durchgeführt. Da es sich um eine Transaktion handelt, wird keiner der beiden Datensätze angepasst, obwohl das Update beim Datensatz '1111' nicht gegen die referentielle Integrrität verstösst.

\subsubsection{Test Delete in Filme}
Wir löschen einen Film, auf welchen keine Referenz verweist:

\begin{lstlisting}
DELETE FROM filme
WHERE katalognr = 2588;
\end{lstlisting}

Es wird keine Exception geworfen. In der Tabelle Filme ist der Datensatz mit der katalognr 2588 nicht mehr vorhanden:

\begin{lstlisting}
 KATALOGNR TITEL                          MINDESTALTER    GEBUEHR
---------- ------------------------------ ------------ ----------
      2028 Casablanca                                9        8.5 
      1245 Ocean's Eleven                           12        9.5 
      2239 A Space Odyssee                          12        7.5 
      1062 Pulp Fiction                             16        8.5 
      1111 Erin Brockovich                           9        8.9 
      2468 Ratatouille                               6        7.5 
\end{lstlisting}

Für den zweiten Fall versuchen wir einen Film zu löschen, zu welchem DVDKopien existieren, auf welchen also eine Referenz verweist:

\begin{lstlisting}
DELETE FROM filme
WHERE katalognr = 2468;
\end{lstlisting}

Es wird eine Exception geworfen von unserem Trigger der die Meldung 'es gibt noch dvdkopien dieses filmes' ausgibt.

\begin{lstlisting}
Fehler beim Start in Zeile 1 in Befehl:
DELETE FROM filme
WHERE katalognr = 2468
Fehlerbericht:
SQL-Fehler: ORA-20000: es gibt noch dvdkopien dieses filmes
ORA-06512: in "MVDBS10.FILME_DELETE", Zeile 11
ORA-04088: Fehler bei der Ausfuehrung von Trigger 'MVDBS10.FILME_DELETE'
ORA-02063: vorherige 3 lines von ORION.HELIOS.FHNW.CH
20000. 00000 -  "%s"
*Cause:    The stored procedure 'raise_application_error'
           was called which causes this error to be generated.
*Action:   Correct the problem as described in the error message or contact
           the application administrator or DBA for more information.
\end{lstlisting}

Es wurden keine Daten in der Tabelle Filme gelöscht:

\begin{lstlisting}
 KATALOGNR TITEL                          MINDESTALTER    GEBUEHR
---------- ------------------------------ ------------ ----------
      2028 Casablanca                                9        8.5 
      1245 Ocean's Eleven                           12        9.5 
      2239 A Space Odyssee                          12        7.5 
      1062 Pulp Fiction                             16        8.5 
      1111 Erin Brockovich                           9        8.9 
      2468 Ratatouille                               6        7.5 
\end{lstlisting}

\end{document}