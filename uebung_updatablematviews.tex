\documentclass[11pt,a4paper,parskip=half]{scrartcl}
\usepackage{ngerman}
\usepackage[utf8]{inputenc}
\usepackage[colorlinks=false,pdfborder={0 0 0}]{hyperref}
\usepackage{graphicx}
\usepackage{caption}
\usepackage{longtable}
\usepackage{float}
\usepackage{textcomp}
\usepackage{geometry}
\usepackage{listings}
 \usepackage{german}
\geometry{a4paper, left=30mm, right=25mm, top=30mm, bottom=35mm} 
\usepackage{listings}
\lstset{breaklines=true, breakatwhitespace=true, basicstyle=\scriptsize, numbers=left}
\title{mvdbs: Updatable Materialized View}
\author{Tobias Lerch, Yanick Eberle, Pascal Schwarz}
\begin{document}
\maketitle

\section{Einleitung}
Ihre Lösung muss aus folgenden Teilen bestehen:

• Eine Beschreibung Ihrer Szenarios


\section{Replikation einrichten}
Die Master Site und die Materialized View Site wurde bereits eingerichtet, somit müssen nur noch die jeweiligen Gruppen erstellt werden.
\subsection{Erstellen der Master Group}
Wir verbinden uns als Benutzer repadmin auf den Telesto Server und führen folgende SQL Statements aus.\\

\lstinputlisting{SQLStatements/01_master_group.txt} 
Das Resultat des SQL Developers ist ein einfaches  \glqq anonymer Block abgeschlossen\grqq. Somit ist die Master Gruppe erstellt.\\

\lstinputlisting{SQLStatements/02_master_repobj.txt}
Die Relation Filialen ist nun ein Replikationsobjekt  und wird der Master Gruppe hinzugefügt. \\

\lstinputlisting{SQLStatements/03_rep_support.txt}
Dieses Statement erstellt die Trigger und Packages, welche für die Replikation gebraucht werden.\\

\lstinputlisting{SQLStatements/04_resume_master.txt}
Die Änderungen werden in den Replikationsprozess aufgenommen.\\

\subsection{Erstellen der Materialized View Group}
Wir verbinden uns als Benutzer mvdbs10 auf den Telesto Server und führen folgende SQL Statements aus.\\

\lstinputlisting{SQLStatements/05_matView.txt}
Auf Telesto wurde nun die Materialized View erstellt und mit \glqq materialized view LOG erstellt.\grqq bestätigt.\\

\lstinputlisting{SQLStatements/06_DBLink.txt}
Der Database Link wird als Benutzer mvdbs10 auf dem Server ganymed erstellt.\\

\lstinputlisting{SQLStatements/07_mview_repgrp.txt}
Dieses Statement erstellt eine neue Materialized View Group.\\

\lstinputlisting{SQLStatements/08_refresh.txt}
Es wird eine Refresh Gruppe erstellt, welche einen stündlichen refresh definiert.\\

\lstinputlisting{SQLStatements/09_mview.txt}
Als Benutzer mvdbs10 auf dem Server ganymed wird die Materialized View erstellt.\\

\lstinputlisting{SQLStatements/10_mview_repobj.txt}
Als Benutzer mviewadmin auf ganymed wird die Relation Filialen als Replikationsobjekt zu der Materialized View Group hinzugefügt.\\

\lstinputlisting{SQLStatements/11_refresh.txt}
Die Materialized View wird zur Refresh Grupppe hinzugefügt.\\

\lstinputlisting{SQLStatements/12_demand_refresh.txt}
Mit diesem Statement kann der Refresh direkt ausgeführt werden.\\

\section{Testszenarien}
\subsection{Ohne Konflikt}
\subsubsection{updates}
Beschreibung:\\
Master Site:\\
	---SQL-Query:\\
	---Log:\\
Materialized View Site:\\
	---SQL-Query:\\
	---Log:\\
\subsubsection{deletes}

\subsubsection{inserts}

\subsection{Mit Konflikt}
\subsubsection{updates}

\subsubsection{deletes}

\subsubsection{inserts}

\subsection{Regel zur Konfliktauflösung}

\subsection{Mit Konflikt und Konfliktauflösung}
\subsubsection{updates}

\subsubsection{deletes}

\subsubsection{inserts}


\end{document}